\chapter{Conclusion}
% summary and directions for future work

\section{Performance Comparison}

In order to evaluate the usefulness of each of these policy types, it is necessary to evaluate the speed with which they cover a wide variety of environments.

In addition to the policies described so far, one more policy is included as a benchmark: the random filtered policy. While all other policies have at least some high level strategy guiding the sequence of motions they command, the random filtered policy only rules out moves that obviously have no use in a coverage task. These moves include hovering in place or attempting to visit locations outside the environment boundaries. Such a policy is obviously not a serious contender for the most efficient way to cover an environment of any meaningful size or complexity. However, including its performance in these results puts the achievements of the other policies into perspective.

Two other policy instances exist in \textit{oprc\_env} besides these. One is the Random Policy, which is similar to the above except that it contains no restrictions whatsoever to the moves commanded at any given time. The other is the PolicyMap, which contains an explicit map of situations to the associated actions to command. For obvious reasons, neither of these policies scales in performance to handle environments of even a modest size. As a result, the performance of these policies is not measured here.

\section{Performance Comparison on Simple Environments}

By design, the performance and relative efficiency of these policies should vary wildly based on what type of environment they explore. To give an initial sense of how these policies behave under somewhat idealized conditions, the following are the scores obtained by each of the policies explored in this work on a large but simple square environment. The environment has dimensions 48x48, and is divided into two equal halves by the amount of scrutiny required to observe its patches.

\begin{table}[h]
\begin{center}
 \begin{tabular}{||c c c c c c c ||}
 \hline
 LSP   & HSFP  & CLP  & LSTP  & CLSTP & ALBP & HSTP \\
 \hline
 23863 & 19982 & 8702 & 23066 & 6778  & 7240 & 5392 \\
 \hline
 \end{tabular}
\end{center}
\caption{Performance Comparison on a Simple Environment}
\end{table}

\subsection{Discussion of Results}

For the most part, these numbers are as expected. The three policies that do almost nothing to take advantage of their access to multiple drones are the Low Sweep Policy (LSP), the High Sweep First Policy (HSFP), and the Low Spanning Tree Policy (LSTP). The average coverage times for each of these policies is significantly larger than it is for any of the multi agent policies.

Among these three policies, the relative sizes of the scores achieved are also as expected. The Low Sweep Policy, being the simplest, performs the worst. Given the restriction that this policy must always fly low to the ground, however, even the LSP is reasonably efficient on this environment. For the most part, any inefficient motion performed by this policy is related to overly aggressive path optimization related to the custom penalty function this policy incorporates into A-Star search. While this penalty does a reasonably good job on a variety of environments, it results in behavior that causes slightly inefficient movement near the borders of this environment.

The Low Spanning Tree Policy performs slightly better, with an improvement of 797 time steps compared to the Low Sweep Policy. This improvement reflects the fact that this policy is able to achieve a nearly optimal path planning performance on such a simple environment.

The High Sweep First Policy (HSFP), performs somewhat better than the previous two because of its initial ability to view the entire environment with low scrutiny in a relatively short time. Because half of the locations in this environment must be viewed with high scrutiny, a substantial portion of the space covered by this policy during its high flying phase must be repeated at a low altitude. In other environments, having to re-cover so much space is often a sign that the environment is a bad fit for these hierarchical policies. However, because all of the high scrutiny locations in this environment are grouped in a large rectangular region, the HSFP is able to cover them very efficiently once it has moved to a low altitude. As a result, the time lost to duplicated work is more than recovered in the time saved by covering the low scrutiny half of the environment faster than usual.

The scores of the multi agent policies are also mostly as expected. The Clustering Low Policy (CLP) is the simplest of these policies, but it still reduces the time taken to explore the environment to just over one third the time taken by the Low Sweep Policy. Because these scenarios were all run with four drones, it is clear that this policy utilizes each of its drones less efficiently than the single agent policies. However, the substantial time saved on overall task performance makes it clear that multi agent coverage algorithms have significant utility whenever the cost of drones is low relative to the importance of coverage speed.

The Clustering Low Spanning Tree Policy (CLSTP) achieves a slightly better score than CLP due to the increased efficiency of the spanning tree paths used to cover each drone's assigned territory. The relative gain in performance seen here is greater than the difference between LSP and LSTP, indicating that some of the lost efficiency exhibited by CLP has been recovered here. 

The Adaptive Low Breadth First Search Policy (ALBP) is the main surprise among these numbers. While the performance of this mores sophisticated algorithm achieves good results on more complex environments, the attempts to adaptively alter each drone's path seems to harm more than it helps when the task being performed is relatively simple.

As expected, the High First Spanning Tree Policy (HSTP) achieves the best results of all on this environment. By combining the multi agent clustering approach and efficient spanning trees used by other policies with a hierarchical search, many of the drones are assigned territory that is mostly or completely low scrutiny locations. These drones are able to complete their tasks in very little time, freeing up extra time to contribute to the high scrutiny portion of the environment. By the time this policy moves to its low altitude phase, the four drones are well distributed in the relevant portion of the environment and are able to complete this task very quickly.

\section{Performance Comparison on Complex Environments}

Of course, very little can be concluded about the usefulness of these results when the only environments analyzed are so simple. In this section, performance on more complex environments is analyzed.

\subsection{Complex Environments where High Scrutiny Dominates}

The following table gives average performance numbers for a series of one hundred environments whose properties make efficient exploration difficult and multiple altitude policies less useful. More specifically, each of these environments was given a complex footprint, and each location in these environments was assigned a scrutiny requirement with an eighty percent chance of high scrutiny. The fact that such a large percentage of environment locations requires high scrutiny is enough to lower the efficiency of high altitude policies dramatically, but the fact that each scrutiny assignment is independent means that these same policies must cover a set of locations that are scattered throughout the environment in an unfriendly way once they move to their low altitude phase.

\begin{table}[h]
\begin{center}
 \begin{tabular}{||c c c c c c c ||}
 \hline
 LSP   & HSFP  & CLP  & LSTP  & CLSTP & ALBP & HSTP \\
 \hline
 22351 & 29523 & 5974 & 17170 & 5078  & 5040 & 6442 \\
 \hline
 \end{tabular}
\end{center}
\caption{Performance on One Hundred Complex Environments Requiring 80\% High Scrutiny}
\end{table}

The complex shapes of the environments explored for this table are generally nowhere close to square in shape. As a result, despite the fact that the overall dimensions of these environments are similar to the simple square environment analyzed in the previous section, these environments tend to contain substantially fewer locations to cover. The fact that that the Low Sweep Policy took an average of 22239 time steps to explore this environment therefore indicates a substantial drop in the average efficiency of drone utilization when exploring these more complex shapes.

The average performance of the High Sweep First Policy demonstrates the inefficiency of exploring this kind of environment with a policy that initially covers the environment from a high altitude. Because the low scrutiny locations in these environments occur only occasionally and in completely random locations, there is very little time saved by the initial high sweep. To cover the whole environment, this policy must spend roughly the same amount of time taken by the Low Sweep Policy \textit{after} it has already viewed the entire space from a high altitude.

The Clustering Low Policy, as before, is the first to show the benefits of multi agent policies. The average coverage time it achieves is therefore substantially faster than with any single agent policy. In addition, the time it takes to cover these environments is substantially lower than when covering a simpler environment from the previous section. This indicates that the CLP is not losing efficiency to the same degree as the Low Sweep Policy when comparing performance on environments with a simple vs complex shape. This superior relative performance is a testament to the usefulness and adaptability of the clustering territory division technique used by this policy.

The Low Spanning Tree Policy achieves substantially better performance on these environments compared to the Low Sweep Policy. This relative performance gain was not seen in the simpler environment's results, indicating that the benefits of the sophisticated path planning technique used by this policy are best utilized in an environment with a complex shape.

Given the results so far, the Clustering Low Spanning Tree Policy performs as expected: A meaningful performance gain from spanning-tree-based path planning is seen on top of the multi agent benefits already realized by the CLP. The Adaptive Low Breadth-First-Search policy is able to improve slightly on this performance. In contrast to its performance on simple environments, this policy seems to realize small opportunities for path optimization from either its breadth-first approach to spanning tree creation or its care in optimizing one drone's path planning based on the territory of the others. However, this performance gain is quite small. As a result, it remains unclear whether this relative difference indicates statistical noise or a real superiority in the strategy used by ALBP.

As expected from environments of this type, the High Spanning Tree Policy performs somewhat worse than its low-flying counterparts. However, the relative performance loss is somewhat better than what was seen between the LSP and the HSFP on these environments. To some extent, this is simply a result of the more efficient path planning of this policy playing out in the high altitude case. However, it is also shows that this policy is able to make small optimizations to its low-altitude paths based on the territory covered from a high altitude.

\subsection{Complex Environments where Low Scrutiny Dominates}

With the above results in mind, it is now worth analyzing cases where high altitude approaches have more opportunities for optimization. The first such case is seen in the following results, in which the relative abundance of high and low scrutiny locations have been swapped compared to the dataset analyzed in the previous section.

\begin{table}[h]
\begin{center}
 \begin{tabular}{||c c c c c c c ||}
 \hline
 LSP   & HSFP  & CLP  & LSTP  & CLSTP & ALBP & HSTP \\
 \hline
 25393 & 24802 & 4942 & 14948 & 4134  & 4050 & 3980 \\
 \hline
 \end{tabular}
\end{center}
\caption{Performance on One Hundred Complex Environments Requiring 20\% High Scrutiny}
\end{table}

For the most part, the low-flying policies in this table require little discussion. Those techniques that use reasonably intelligent path planning do slightly better on these environments, while the Low Sweep Policy does slightly worse. The exact reasons for this are not clear, but can be reasonable attributed to slightly more complex and smaller environments in this dataset. However, ALBP makes a slightly more significant gain in average performance relative to CLSTP on these environments. This provides more evidence that it uses a meaningfully better approach.

The two high altitude policies give more interesting results in this section. The High Sweep First Policy achieves better performance than LSP on these environments. This improvement is marginal, but it demonstrates that some opportunities for path planning optimization can occur after a quick high altitude sweeping phase. These optimizations are still somewhat few due to the erratic distribution of low-scrutiny locations in these environments. With that said, the deterministic nature of these policies makes it easy to conclude that a real performance has been achieved without worrying about the (nonexistent) choice of random numbers leading to the results.

The High Spanning Tree Policy achieves better results than any other policies on these environments. As expected, environments in which low scrutiny locations dominate lead to more opportunities for this kind of policy to achieve good results.

\subsection{Complex Environments where High Scrutiny Patches Appear in Large Clusters}

There are a wide variety of assumptions that could be asserted about the nature of real-world environments that correspond to the abstractions used in these environments. One such assumption is that locations which merit a detailed observation tend to be a relatively small portion of the overall region being examined, and that these more interesting locations tend to occur in relatively large clusters. While this assumption does not apply to every potential application of robotic coverage, it certainly applies to some of them. This model underlies the design of many of the coverage policies compared in this work, as well as some of the environment generation tools that create the worlds these policies explore. The following results analyze how various policies would handle such an environment.

\begin{table}[h]
\begin{center}
 \begin{tabular}{||c c c c c c c ||}
 \hline
 LSP   & HSFP  & CLP  & LSTP  & CLSTP & ALBP & HSTP \\
 \hline
 18097 & 10570  & 4924 & 14352 & 4064  & 4148 & 2728 \\
 \hline
 \end{tabular}
\end{center}
\caption{Performance on One Hundred Complex Environments in which High Scrutiny Locations Appear in Infrequent Clusters}
\end{table}

In this case, the results for all of the low flying policies yield no new insights. The High Sweep First Policy performs surprisingly well, achieving an average performance that takes a little over half as long as its low-flying counterpart.

The High Spanning Tree Policy, achieving the fastest coverage of these environments by a large margin, shows how well suited this strategy is to covering the kind of environment described above. In fact, the combination of hierarchical coverage and intelligent path planning it employs leads to a result that is hard to achieve in other environments: comparable per-drone efficiency to a single agent policy. Since the High Spanning Tree Policy effectively uses four times as many drones as its closest single agent counterpart, HSFP, a naive estimate of the relative time taken between HSFP and HSTP might be 4:1. In fact, most environments are such that the multi agent policies are not superior to the extent that reasoning like this would suggest. This can be attributed to a variety of causes, but they are best summarized as drones needing to move through territory assigned to the others. However, in this set of environments, HSTP achieves a performance ratio of 3.87:1.

If the environments used in this dataset can be assumed to be representative of real world environments in which this kind of coverage task would need to be performed, the performance of HSTP on these policies is quite good. By taking advantage of a new abstraction of the coverage task, variable scrutiny, this relatively unoptimized policy achieves multi agent coverage on a complex environment with per-drone efficiency comparable to reasonable single-agent approaches.

\subsection{Drone Dropout Performance Comparison}

As discussed in previous chapters, the environment developed to simulate coverage scenarios in this work supports a mode of operation with drone failure. In the following table, average performance statistics are computed for the same environments as in the previous table. However, drones in this case were given an 0.05\% chance to fail at each time step. Note that this scenario will not cause a drone to fail if it is the only working drone left in a scenario. While the specific model of drone failure and choice of parameters used here is probably unrealistic, this setup allows for for a quick first look at the implications of each policy design for a case in which drones can fail. The same dataset was used here because these results are best understood in terms of how the introduction of dropout affects each policy individually.

\begin{table}[h]
\begin{center}
 \begin{tabular}{||c c c c c c c ||}
 \hline
 LSP   & HSFP  & CLP   & LSTP  & CLSTP & ALBP & HSTP \\
 \hline
 18097 & 10570 & 11511 & 14352 & 9059  & 8822 & 4866 \\
 \hline
 \end{tabular}
\end{center}
\caption{Performance on Environments With Drone Failure}
\end{table}

It is immediately clear that the three single agent policies have an average performance that is totally unchanged from the previous results. Because the same set of environments was used to test these deterministic policies, it is as expected that these numbers remain the same. Note that while the idea that at least one drone will survive through the entire policy is not a realistic assumption when placed against the drone failure rate used here, but is probably realistic in scenarios with more drones or a lower failure rate. In this very limited sense, these single agent policies are therefore more robust when run with multiple drones.

The other four policies, organized to make full utilization of the number of drones they command, are clearly affected by the introduction of dropout. Three of these policies (CLP, CLSTP, ALBP) performed comparably. Each of these achieved slightly worse than a 2x increase in average coverage time in the presence of drone dropout. The High Spanning Tree Policy, on the other hand, requires a little less than twice its previous average time to cover these environments with dropout. The only explanation offered for this difference is that this policy performs coverage more efficiently than the other three on this set of environments. As a result, it achieves a larger fraction of environment coverage before a drone failure occurs, on average. This means that once one or more drones have failed, a relatively small fraction of the original coverage task remains to be performed less efficiently by the remaining drones. While it is not possible to understand the way these policies adapt to drone failure from these numbers alone, visualizations of individual scenarios similar to this one show that the clustering technique used by these policies to divide territory does a god job at adapting quickly to each drone failure.

\section{Future Work}

Although this work is fairly permissive of complex environment shapes, it does not consider coverage of environment that cannot be exactly represented as a square grid. As noted previously, the search-and-rescue scenarios that motivate this work could benefit from close inspection of tight corners and areas near the perimeter of environment boundaries. Since these boundaries may be due to collapsed or naturally occurring structures, their borders are unlikely to be aligned with cardinal directions. An algorithm that could adapt its behavior near these regions could be of great practical importance. For an example of other work that has achieved this kind of extension in the past, consider the exact coverage variant of STC as described in \cite{STC}. It may also be useful to represent the borders of regions requiring low or high scrutiny as arbitrary polygons, although the opportunity for improved algorithmic performance or other practical benefits are less clear in this case.

Another potential improvement of this work would allow for persistent coverage. Targets may be mobile in certain search and rescue situations, and so it is necessary to re-visit locations periodically in order to make sure nobody has moved into them. %https://ieeexplore.ieee.org/stamp/stamp.jsp?tp=&arnumber=7989156 -- some work on this subject

%Take out this section?
\subsection{Persistent (Learning) Policies}

In the last chapter, it was proved that the optimal policy with which to perform coverage depends on the distribution environments that policy will be asked to solve. In that chapter, this problem was approached by hand crafting policies to perform as well as possible on a complex distribution of environments through experimentation and tweaking of parameters. In the software package developed for this research, \textit{oprc\_env}, the \textit{PersistentPolicy} typeclass allows for policies that can initialize themselves differently with each new scenario they attempt. This makes it possible for these policies to learn strategies for environment exploration that are tweaked over the long term in response to experience. One of the most promising applications of this could be policies that can intelligently swap back and forth between the altitudes they fly at in response to observations and predictions about the content of the environment. While some effort was made to implement such a policy for this work, the results were not useful. It remains to be seen whether such a policy can achieve superior behavior on the kinds of environments explored here.

%In this section, some policy implementations are developed that can recreate a version of this experimentation and tweaking across multiple scenarios. Importantly, these policies perform these tweaks \textit{automatically}, using the PersistentPolicy typeclass and associated environment interface tools to refine an approach specifically tuned to a distribution of environments encountered online.

It's worth addressing why this online learning behavior could be superior to a hand crafted approach. The distribution of real world environments in which you might want to deploy drone based robotic coverage is not at all easy to characterize. Furthermore, practical applications of this kind of  technology may see it deployed to various owners around the world. Any one of these teams of robots is likely to be deployed in environments with somewhat different properties the environments seen by other teams. As a result, there may be an opportunity to optimize the behavior of each of these teams of robots to the particular circumstances in which it is deployed. Such optimizations would not be easy to achieve through any sort of manual effort, so a well crafted policy that is capable of changing itself between individual operations may be the best way to achieve this optimization. In addition, there is some theoretical value in seeing how coverage policies can be automatically optimized for an experimentally determined distribution of environments

\section{Final Remarks}

The policies compared in this chapter demonstrate that there are several ways to design a coverage strategy. The particular strategies shown off here have trade offs in their ability to adapt to the use of multiple drones, efficiency of path planning, ability to use hierarchical coverage strategies, and simplicity of design. Each of these considerations has meaningful implications for the potential uses of similar policies in a real-world application of robotic coverage. One of the primary insights from the analysis in this chapter is that the exact distribution of environments for which coverage is needed has overwhelming implications for the relative performance of various policies, and therefore this information should inform the design of the coverage policies designed to work in such an application. To the extent that environments requiring various levels of scrutiny during a coverage task are realistic, this work demonstrates that policies which take advantage of this novel problem formulation could achieve superior performance on such environments.