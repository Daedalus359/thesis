\chapter{Conclusion}
% summary and directions for future work

\section{Performance Comparison}

In order to evaluate the usefulness of each of these policy types, it is necessary to evaluate the speed with which they cover a wide variety of environments.

In addition to the policies described so far, one more policy is included as a benchmark: the random filtered policy. While all other policies have at least some high level strategy guiding the sequence of motions they command, the random filtered policy only rules out moves that obviously have no use in a coverage task. These moves include hovering in place or attempting to visit locations outside the environment boundaries. Such a policy is obviously not a serious contender for the most efficient way to cover an environment of any meaningful size or complexity. However, including its performance in these results puts the achievements of the other policies into perspective.

Two other policy instances exist in \textit{oprc\_env} besides these. One is the Random Policy, which is similar to the above except that it contains no restrictions whatsoever to the moves commanded at any given time. The other is the PolicyMap, which contains an explicit map of situations to the associated actions to command. For obvious reasons, neither of these policies scales in performance to handle environments of even a modest size. As a result, the performance of these policies is not measured here.

\section{Performance Comparison on Simple Environments}

By design, the performance and relative efficiency of these policies should vary wildly based on what type of environment they explore. To give an initial sense of how these policies behave under somewhat idealized conditions, the following are the scores obtained by each of the policies explored in this work on a large but simple square environment. The environment has dimensions 48x48, and is divided into two equal halves by the amount of srutiny required to observe its patches.

\begin{center}
 \begin{tabular}{||c c c c c c c ||}
 \hline
 LSP   & HSFP  & CLP & LSTP & CLSTP & ALBP & HSTP \\
 \hline
 23863 & 19982 & 8702 & 23066 & 0 & 0 & 0 \\
 \hline
 \end{tabular}
\end{center}

\section{Performance Comparison on Complex Environments}

\section{Drone Dropout Performance Comparison}

\section{Efficiency of Drone Utilization}

%dropout makes the prev chapter policies multi agent

%Take out this section?
\section{Persistent (Learning) Policies}

In the last chapter, it was proved that the optimal policy with which to perform coverage depends on the distribution environments that policy will be asked to solve. In that chapter, this problem was approached by hand crafting policies to perform as well as possible on a complex distribution of environments through experimentation and tweaking of parameters. In the software package developed for this research, \textit{oprc\_env}, the \textit{PersistentPolicy} typeclass allows for policies that can initialize themselves differently with each new scenario they attempt. This makes it possible for these policies to learn strategies for environment exploration that are tweaked over the long term in repsonse to experience. One of the most promising applications of this could be policies that can intelligently swap back and forth between the altitudes they fly at in response to observations and predictions about the content of the environment. While some effort was made to implement such a policy for this work, the results were not useful. It remains to be seen whether such a policy can achieve superior behavior on the kinds of environments explored here.

%In this section, some policy implementations are developed that can recreate a version of this experimentation and tweaking across multiple scenarios. Importantly, these policies perform these tweaks \textit{automatically}, using the PersistentPolicy typeclass and associated environment interface tools to refine an approach specifically tuned to a distribution of environments encountered online.

It's worth addressing why this online learning behavior could be superior to a hand crafted approach. The distribution of real world environments in which you might want to deploy drone based robotic coverage is not at all easy to characterize. Furthermore, practical applications of this kind of  technology may see it deployed to various owners around the world. Any one of these teams of robots is likely to be deployed in environments with somewhat different properties the environments seen by other teams. As a result, there may be an opportunity to optimize the behavior of each of these teams of robots to the particular circumstances in which it is deployed. Such optimizations would not be easy to achieve through any sort of manual effort, so a well crafted policy that is capable of changing itself between individual operations may be the best way to achieve this optimization. In addition, there is some theoretical value in seeing how coverage policies can be automatically optimized for an experimentally determined distribution of environments.

\section{Future Work}

Although this work is fairly permissive of complex environment shapes, it does not consider coverage of environment that can't be exactly represented as a square grid. As noted previously, the search-and-rescue scenarios that motivate this work could benefit from close inspection of tight corners and areas near the perimeter of environment boundaries. Since these boundaries may be due to collapsed or naturally occurring structures, their borders are unlikely to be aligned with cardinal directions. An algorithm that could adapt its behavior near these regions could be of great practical importance. For an example of other work that has achieved this kind of extension in the past, consider the exact coverage variant of STC as described in \cite{STC}. It may also be useful to represent the borders of regions requiring low or high scrutiny as arbitrary polygons, although the opportunity for improved algorithmic performance or other practical benefits are less clear in this case.

Another potential improvement of this work would allow for persistent coverage. Targets may be mobile in certain search and rescue situations, and so it is necessary to re-visit locations periodically in order to make sure nobody has moved into them. %https://ieeexplore.ieee.org/stamp/stamp.jsp?tp=&arnumber=7989156 -- some work on this subject