\chapter{Introduction}
% 20 pages or so
% rules of the game
% objective of my algorithms
% literature review

\section{Robotic Coverage Background}

Robotic motion planning problems are typically concerned with finding an achievable path between two robot configurations. These configurations could be the current location and desired destination of a robotic vehicle, the current pose of a robotic arm and the pose required for it to grasp an object, or any other pair of robot states which correspond to ``start" and ``goal" configurations. \textit{Robotic coverage} problems are another kind of motion planning problem. The goal of a robotic coverage problem is to find a path for the robot to follow, during which it will visit every location in a region. This region could be an area of floor to be vacuumed, the surface of a car body that needs to be painted, or the interior of a building that needs to be mapped. In all of these cases, the task is complete only once the entire region has been visited, or \textit{covered} by a robot. Usually, another goal of these problems is to achieve coverage in a way that minimizes the time or number of movements required to complete the task.
The robotic coverage problem can be viewed as a continuous generalization of the travelling salesman problem [7]. Thus, any sufficiently general instance of the problem is NP-Hard. 

% based on [1], section 1. Is this too similar to the original?
%should I elaborate on the early heuristic algorithms that did not guarantee coverage
One particularly well studied type of robotic coverage problem is the case where the coverage region is some subset of the plane, and the robots are mobile robots that can only move in the plane. A good overview of this kind of coverage problem can be found in [1]. Algorithms to perform robot coverage in the plane have been studied since the late 1980's. Early algorithms for this purpose relied on the use of heuristics, and these algorithms usually could not guarantee that complete coverage would always be achieved. By around 2000, planar coverage algorithms were often able to guarantee complete coverage. A completeness guarantee is often proved by showing that the algorithm breaks down the coverage space into smaller regions that are relatively trivial to cover completely. Coverage algorithms that employ this kind of technique are known as \textit{cellular decomposition} algorithms.

One type of cellular decomposition fits a square grid to the coverage space. Because a general arbitrary set in the plane can not be exactly expressed as a union of grid-aligned square regions, this technique is often called approximate cellular decomposition. Typically, each cell in this kind of decomposition can be covered entirely by a single robot configuration. For example, consider dividing a grass lawn into squares for a robotic lawn mowing task. If each square in the cellular decomposition is small enough, a lawnmower whose center is aligned with a square's center will have mowed that entire square of the lawn. In coverage problems where this assumption applies, proving that a coverage path visits each cell in the approximate decomposition is sufficient to guarantee complete coverage.

In addition to approximate cellular decomposition, there are also techniques that decompose a coverage space into a more diverse collection of cell shapes. Examples of such shapes include cells of fixed with whose top and bottom boundaries can vary in shape [3], or trapezoidal cells whose parallel sides all run in the same direction [4]. Depending on whether the union of these cells is exactly equal to the coverage space, such techniques are called either \textit{semi-approximate} or \textit{exact} cell decompositions. In the case of trapezoidal cell decomposition, it is possible to combine the trapezoidal cells into a smaller number of large cells, each of which are able to be covered by a simple back-and-forth sweeping motion. This approach is known as a \textit{boustrophedon} decomposition [4].

\section{Spanning-Tree based Coverage Algorithms}

When coverage is performed by a square shaped tool attached to a mobile robot, and when the coverage area can be exactly decomposed into a grid of squares which have double the side length of the robot tool, it is possible to compute an optimal covering path in linear time using an approach based on spanning trees [2]. The authors developed several variants of this approach with different time, memory, and prior knowledge requirements. In all cases, it was assumed that a mobile robot would complete the coverage task with a square shaped tool of size \textit{D}. It was also assumed that the tool could only move in the four cardinal directions when completing the task. Finally, it was assumed that the region to be covered could be approximated by cells in a grid of squares with size 2\textit{D}. The authors argue that the final assumption is justified for realistic environments in which the size of the robots tool is significantly smaller than the dimensions of the area to be covered. Under these assumptions, the \textit{Spanning Tree Covering} (STC) algorithm is able to generate a path through the space which visits each grid-aligned square cell of size D exactly once.

The offline variant of STC works by representing the coverage area as a graph \textit{G}. Nodes on this graph represent the center of a square with size 2\textit{D}, and the set of nodes corresponds to the set of full cells when a grid with square size 2\textit{D} is overlaid on the coverage space. Edges exist between any two nodes whose corresponding cells are adjacent in the coverage space. As the name suggests, STC's key step is the creation of a spanning tree on \textit{G}, using any node in \textit{G} as the root of the tree. Many algorithms exist to compute spanning trees, but two popular examples are closely based on breadth-first-search and depth-first-search [9].

Once this spanning tree has been created, the coverage area is subdivided into squares of size \textit{D}, such that each node of \textit{G} coincides with the corners of four different squares in the subdivided map. Starting at any one of these squares, each of which is exactly the shape and size of the robot's coverage tool, the robot can simply move counterclockwise around the spanning tree as if it were performing wall following. In this way, each cell in the coverage space will be visited exactly once by the robot. Because of this property along with the fact that the path generated by STC end up adjacent to where it started, the path found by STC can be considered a Hamiltonian cycle on \textit{G}. Although constructing a Hamiltonian path on a general planar grid is NP-complete [10], the assumed properties of \textit{G} make it practical to compute such a path on this graph in linear time. It is also worth noting that this coverage path's start and end positions will be in adjacent cells. This can be beneficial in practice, as it allows deployment and retrieval of the robot to happen in the same place.

%insert a picture here

This algorithm has a few interesting variants. The first runs online, and effectively circumnavigates a depth first search spanning tree that it constructs on the fly [2]. Restriction to use of depth first search is the primary disadvantage of this approach compared to offline STC. While the coverage paths based on depth first search are equivalent to those created through any other method in terms of final path length, other spanning tree algorithms have practical benefits. For example, spanning tree algorithms such as Prim's algorithm or Kruskal's algorithm can create minimal spanning trees on weighted graphs, and this can be exploted to create coverage paths that minimze turning. In later work, the authors of STC do exactly that in an algorithm called Scan-STC, which also makes additional improvements on the original STC as discussed below. For details on Prim's algorithm and Kruskal's algorithm, refer to [9]. 

The second online variant of the STC algorithm is described by its authors as \textit{ant-like}. In this case, the robot must be able to mark visited sub-cells as visited through some means such as leaving pebbles on the ground. In addition, the robot must be able to sense the presence of these makers in the nodes of \textit{G} adjacent to its current position. By marking each visited sub cell as the robot leaves that cell, the robot is able to proceed on the same path it would have created in the depth first search based online approach. Because this algorithm uses markers to identify visited nodes, it requires only O(1) memory. However, it requires the use of O(N) markers, where N is the number of sub-cells in the coverage area. Nonetheless, this algorithm presents interesting ideas for multi-agent approaches and prevention of drift in pose estimation. In both of these cases, physical and immobile records of a robot's past location can inform the decision making of any robot currently near some of those records.

Finally, there is a variant of STC that achieves \textit{exact} coverage of an environment, even when that environment can not be represented by completely filled cells on a grid of squares with size 2\textit{D}. However, this approach still requires that the connectivity of the environment is represented adequately by such a grid. The resulting algorithm simply notices any cells that are partially occupied by a previously unseen obstacle when it passes next to them in the course of normal online STC. When such a cell is encountered, a neighborhood is swept around the entire obstacle. Then, the robot returns to traversing the fully in-bounds cells according to its usual procedure. For details concerning the size of this neighborhood and the correctness of the resulting algorithm, see [2].

In the previously mentioned Scan-STC algorithm, STC is improved to explicitly handle boundary cells, or graph cells that are partially occupied by an obstacle. Using the same notation as before, this algorithm generates a path with length no greater than $ (n + m)\textit{D} $, where n is the number of nodes in \textit{G}, and m is the number of boundary cells [8]. The authors assume that no node will contain an arrangement of obscales such that the graph of that node's subnodes is disconnected. If an obstacle exists such that the normal grid of nodes would violate this assumption, the corresponding location in the coverage area is assigned two different nodes. With that assumption in place, avoidance of obstacles is done by simply displacing the normal path that goes around the obstacle. This displacement tactic is what leads to the m term appearing in the maximum path length bound. The authors claim that in practical environments, the path length is closer to $ n + \frac{m}{2} $.

%%illustrate an example of a node whose subnodes are disconnected (2 out of bounds non-adjacent subcells)

The other significant insight of Scan-STC relates to its name. By allowing the mobile robot to sense the state of the eight cells the surround its current position, it is possible to discover that horizontal neighbors of the current cell has a free vertical neighbor, and that the current cell has a free vertical neighbor in the same direction. In this case, it is guaranteed that a path into the nodes horizontal neighbor exists such that the vertical neighbors of these cells appear between the current cell and its horizontal neighbor. With this insight, it is possible to skip the creation of a horizontal edge between the current node and its horizontal neighbor. The benefit of this approach is that it leads to coverage paths that move primarily in the vertical directions. This can be useful for limiting the amount of turns performed, thereby reducing the number of opporuntities to introduce dead reckoning error from imprecise turns.

\section{Muti-Agent Robotic Coverage}

Another imporant feature of a robotic coverage algorithm is whether that algorithm controls a single robot or a group. Algorithms that control a group of robots for this purpose are known as multi-agent coverage algorithms. Multi agent approaches to coverage can often complete a coverage task several times faster than a single agent approach because of the division of work that use of multiple robots enables. Use of multiple robots can enhance robotic coverage by allowing for more precise localization through information sharing. Finally, multi-agent methods are often able to adapt to the failure of one or more robots, as there may be remaining robots capable of completing the coverage task [1]. However, multi-agent techniques often require more complex motion planning strategies to coordinate the motion of multiple robots in a way that takes full advantage of these potential efficiency improvements and other benefits.

It is also possible to adapt a spanning tree coverage algorithm to the multi-agent case [6][11]. In their work on this subject, Noam Hazon and Gal Kaminka generalize the STC algorithm to the multi agent case, resulting in an algorithm called \textit{Multirobot Spanning-Tree Coverage}, or MSTC. The MSTC problem assumes that the coverage area can be adequately approximated by a grid of square cells with side length 2\textit{D}, as in the spanning tree coverage algorithm. As with the original version of STC, it is assumed that all cells in the coverage space are free of obstacles. It is also assumed that all robots have the same speed and tool size.

With these assumptions in place, the first version of MSTC simply constructs a spanning tree of the space offline, then has all of the drones start to follow the spanning tree from their initial positions, much the same as single agent STC. If the robots are able to share information about their status, i.e. working or broken, and whether they have completed their assigned section, then it is possible to have all working robots continue to follow the spanning tree path until the coverage task is complete. Under these assumptions, complete coverage is guaranteed as long as one robot remains in working order until the end of the task.

If there are k robots exploring a coverage space with n sub-cells as described above, then the best case time for this algorithm is shown to be $ \frac{n}{k} - 1 $. This performance is achieved when the starting points of the robots are distributed such that each robot is evenly spaced along the length of the spanning tree coverage path. The worst case is stated to be $ n - k - 1$, for the case where all k robots start on k adjacent subcells that appear together in the spanning tree graph. Implicit in both bounds is the assumption that no robots fail. In a pathological case with the same initial conditions used to give the previously stated worst case bound, the $ k - 1$ robots that start directly in front of the last robot could all fail at time $ n - k - 2$, just before the front robot reached the back robot's position. In this case, it would take another $n - 1$  steps for the remaining robot to complete the coverage task according to Algorithm 2 as described by the authors, leading to a total coverage time of $ 2 n - k - 2 $. The authors note that many practical multi-robot coverage scenarios require starting all robots from the same location, meaning that performance close to the worst case bound $ n - k - 1 $ is relatively common. This motivates the development of a more sophisticated coverage algorithm. By assuming that backtracking is allowed, the authors develop an algorithm with a worst case of $ \frac{n}{2} - 1 $ for $ k > 2 $. However, because all algorithms in [11] only consider the possibility of moving foreward and backward along the initially calculated spanning tree path, many opporunities for further improvement are missed.

Work by Zheng et al. surpasses the average-case performance of MSTC with their \textit{Multi-Robot Forest Coverage} (MFC) algorithm [6]. As the name suggests, MFC computes a separate spanning tree for each of the robots working on the coverage task. For convenience, the authors of MFC also assume that multiple robots can occupy the same subcell simultaneously, and that the coverage area can be represented by grid-aligned cells with side length twice that of the robot's coverage tool. Experimental results showed that MFC achieved performance close to a theoretical lower bound in many cases. In addition, the authors of [6] were able to prove that their algorithm's cover times are no worse than eight times the optimal value, thanks to properties of the tree cover algorithm for general graphs which they adapted to this purpose.

\section{Problem Statement}

As with much of the previously discussed work, this work presents multi-agent coverage algorithms based on spanning trees. The details of the coverage problem stated here vary somewhat from other work - these differences are motivated by a use case in which a team of aerial drones (UAVs, robots) must perform sensor coverage of an area affected by disaster. Two particularly novel features are introduced through this consideration: robots that are able to change the size of their field of view, and environments with subsections which require different levels of scrutiny. These two changes are closely related to one another.

Concrete motivation for an environment that requires varying levels of scrutiny is based on the idea that not all areas are worth viewing in great detail when gathering information for a search and rescue effort. For example, some areas of the coverage environment may be covered in water or similarly uniform. Such areas may not be worth covering in detail because their uniform nature makes it easy to conclude that nothing of interest is happening in these regions. Oppositely, trying to observe a region with dense tree cover may not be practical in a coverage scenario. In contrast, areas with partial occlusions, areas with visually complex damaged structures, or places where people are particularly likely to be in need of help may merit closer looks. Another assumption of this problem is that the distribution of locations requiring one level of scrutiny or another is not known in advance of coverage. It may not be possible to know the exact location of collapsed structures, flooded regions, or other environment features that may come up with relatively little warning in an emergency.

The other significant novel feature of this problem formulation, robots with a variable field of view, arises naturally in light of the above consideration. Real drones can usually alter the size of a ground area being photographed. Increasing the size of the observation area may be achieved by ascending with a constant viewing angle or by using a camera zoom feature to increase the viewing angle. Likewise, decreasing the field of view may be achieved by descending, zooming in, or by some combination of these. Increasing the field of view has the obvious benefit of allowing a robot to cover more of its environment at once. Decreasing the field of view increases the detail with which a drone observes the area directly below it, and this allows high detail viewing of those areas that require it. While the distribution of environment sections that require low vs high scrutiny is unknown a priori, it is assumed that the boundaries of the coverage area and of the obstacles that border it are known in advance.

%go into specifics about the FOV sizes and levels of scrutiny I use here!!!

Unlike most other work on spanning-tree based coverage, this work assumes that the robots can move along diagonals, rather than only in the four directions orthogonal to the sides of its tool. For convenience of notation, the four directions commonly used by spanning tree robots will be referred to as \textit{cardinal directions}, and the four directions that are offset by $\frac{1}{8}$ turn from the cardinal directions will be called intercardinal directions. In addition, concrete examples shown in this work may refer to a particular cardinal direction such as East, or a particular intercardinal direction such as South-West (SW).

One reasonable concern about the introduction of diagonal motion in a world represented by an approximate cellular decomposition is how it affects the connectivity of the coverage area. This work assumes that all motions in the environment move directly between the centers of the cells created by approximate cellular decomposition. In addition, it is assumed that the robots are able to move directly between cells which share only a corner, even if one or both of the adjacent cells that would connect them are out of bounds. Since other work typically assumes that the robot is roughly the same size as the area it covers, this assumption requires justification. Because the problem explored here is motivated by a team of autonomous drones observing an area from above, it is safe to assume that the robots are significantly smaller than the area they are covering at a particular moment. As a result, it seems justifiable to assume that the robots are able to move between diagonally connected in bounds cells, as there is likely to be enough room for a drone to complete such a move in a realistic scenario.

%attach figure illustrating the connectivity of a 2x2 with only two diagonally opposite patches in bound

In this work, the author assumes that all robots start out in the same location. This assumption makes the work less general than other work that allows for arbitrary starting locations, such as [11]. However, starting all robots in one location is often a practical assumption, as it allows for easy deployment of robots from a single location. It is also worth noting that starting each robot in the same place often makes for a more challenging coverage task than starting with robots distributed throughout the space. This is because spatially distributed robots may already be close to evenly dispersed along a single Hamiltonian cycle through the coverage space, and so moving to this extremely convenient state may be a low cost operation. Note that in order to take advantage of this situation with the highest probability, robots must be able to move freely between their starting position and an idealized starting postion for spanning tree following. Note that this initializtion step is not addressed by the MSTC algorithm in [11].

Unlike many of the efforts to adapt spanning tree coverage to a multi agent case, this work allows for environments that can't be cleanly represented as a collection of obstacle-free cells twice the size of a robot's coverage footprint. This assumption complicates the issue of creating and traversing spanning trees significantly. However, it is likely to be an important practical assumption for many realistic tasks. For example, in a search and rescue setting, people may be trapped around collapsed structures and other difficult to navigate corners of the coverage area. The capaility to plan spanning tree paths through such complex environments is also useful for the case where a robot has viewed an area with low scrutiny and must now revisit a small portion of that area that requires high scrutiny. Although it is not necessary to forbid revisiting locations that have already been adequately covered, a good path planning strategy should not visit these locations unless doing so is the most efficient way of covering the locations that do require a visit.

Like much of the work that addresses multi-agent robot coverage, this work assumes that the robots share information while performing coverage. This assumption abstracts away a significant and complex barrier to implementing the described algorithms on real robots. However, because this work is motivated by a use case in which the robots are substantially smaller than their coverage area, and because an approximate cellular decomposition is used to describe the set of locations in the coverage area, it is possible to communicate full information between the robots in this scenario using only small and relatively infrequent mesages. In addition, realistic flying robots are almost always equipped with long range communication devices to allow manual control by a human operator and to transmit observations in real time. For a search and rescue scenario in particular, it seems likely that the drones would be equipped with the ability to transmit high resolution photos at the same rate that coverage information is passed, meaning that the passing key planning information represents only a small fraction of the drone's available bandwidthe. With the considerations in mind, it seems safe to assume that the required level of communication between drones is at least possible in a realistic scenario. 

As previously stated, each drone in this formulation has access to moves that increase and decrease its field of view. For simplicity, it is assumed that a drone's field of view can be in only two states which correspond to two possible altitudes. A drone flying low can view a single cell in the environment's approximate cellular decomposition. A drone flying high can view a 3x3 grid of these environment cells. Depending on which altitude the drone currently occupies, it has access one of two vertical motions: ascend and descend. Because this problem statement allows for more types of robot state changes than the previously covered work, it is worth addressing the optimization criteria and cost function for this task. Much of the existing work on multi agent coverage aims to find a set of robot paths that complete the coverage task while minimizing the length of the longest path taken by any of the robots. In this work, it makes more sense to assign time costs to all of the moves available to the drones. Then, the optimization goal for this task is to achieve complete coverage in the least possible time. It is assumed that ascending and descending each take as long as a cardinal motion between adjacent environment cells. Intercardinal motion between cells that share a vertex is assumed to take a factor of $ \sqrt{2} $ more time than cardinal motions in order to correspond with the greater euclidean distance between the centers of these cells.
