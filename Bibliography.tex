\chapter{Bibliography}

% [#], Author, "Title," \textit{publication name,} vol. ##, PubMonth, PubYear.

\noindent [1] H. Choset, ``Coverage for robotics - A survey of recent results,'' \textit{Annals of \mbox{Mathematics} and Artificial Intelligence,} vol. 31, October 2001.
%Issue 1-4, pp 113-126
	% complete a priori environment information or sensir based coverage, see section 1
	% random approaches, see section 2
	% potential field methods, sec 3
	% semi-approximate cellular decomposition, sec 4

\noindent [2] Y. Gabriely and E. Rimon, ``Spanning-tree based coverage of continuous areas by a mobile robot,'' \textit{Annals of \mbox{Mathematics} and Artificial Intelligence,} vol. 31, October 2001.
%Issue 1-4, pp 77-98

\noindent [3] S. Hert, S. Tiwari, and V. Lumelsky, ``A terrain-covering algorithm for an AUV,'' \textit{Autonomous Robots,} vol. 3, June 1996.
%Issue 2-3, pp 91-119
	%an algorithm that can be performed on-line

\noindent [4] H. Choset, E. Acar, A. Rizzi, and J. Luntz, ``Exact Cellular Decomposition in Terms of Critical Points of Morse Functions,'' \textit{Proceedings of the 2000 IEEE International Conference on Robotics and Automation,} April 2000.

\noindent [5] D. Kurabayashi, J. Ota, T. Arai, S. Ichikawa, S. Koga, H. Asama, and I. Endo, ``Cooperative Sweeping by Multiple Mobile Robots with Relocating Portable Obstacles," \textit{Proceedings of the IEEE/RSJ International Conference on Intelligent Robots and Systems,} November 1996

\noindent [6] X. Zheng, S. Jain, S. Koenig, and D. Kempe, ``Multi-Robot Forest Coverage," \textit{IEEE/RSJ International Conference on Intelligent Robots and Systems,} August 2005

%alternate citation which may prove what I need, since I can't find a copy of this one:
  %E.M. Arkin and R. Hassin, Approximation algorithms for the geometric covering salesman problem, Discrete Appl. Math. 55 (1994) 197–218.
\noindent [7] E. Arkin, S. Fekete, and J. Mitchell, ``The lawnmower problem," \textit{Proceedings of the 5th Canadian Conference on Computational Geometry,} August 1993
% pp 461-466

\noindent [8] Y. Gabriely and E. Rimon, ``On-Line Coverage of Grid Environments by a Mobile Robot," \textit{Computational Geometry,} April 2003
% pp 197-224

%introduction to algorithms BOOK
\noindent [9] T. Cormen, C. Lieserson, R. Rivest, and C. Stein, \textit{Introduction to Algorithms, Second Edition,} MIT Press and McGraw-Hill, 2001

\noindent [10] A. Itai, C. Papadimitrou, and J. Szwarcfiter, ``Hamilton Paths in Grid Graphs," \textit{SIAM Journal on Computing,}  November 1982

\noindent [11] N. Hazon and G. Kaminka, ``Redundancy, Efficiency and Robustness in Multi-Robot Coverage," \textit{Proceedings of the 2005 IEEE International Conference on Robotics and Automation,} April 2005

%not sure about the format on this one
\noindent [12] R. Durstenfeld, ``Algorithm 235: Random Permutation," \textit{Communications of the ACM,} July 1964

\noindent [13] S. Thrun, ``Robotic Mapping: A Survey," \textit{Exploring Artificial Intelligence in the New Millenium,} 2003
%https://dl.acm.org/citation.cfm?id=779343&picked=prox
%@book{Lakemeyer:2003:EAI:779343,
% editor = {Lakemeyer, Gerhard and Nebel, Bernhard},
% title = {Exploring Artificial Intelligence in the New Millennium},
% year = {2003},
% isbn = {1-55860-811-7},
% publisher = {Morgan Kaufmann Publishers Inc.},
% address = {San Francisco, CA, USA},
%}

%More references to use
  %1
	%S.V. Spires and S.Y. Goldsmith, Exhaustive geographic search with mobile robots along space-filling
	%curves, in: Collective Robotics; Proceedings of the First International Workshop, CRW’98. Paris,
	%France, July 1998, Lecture Notes in Artificial Intelligence, Vol. 1456 (Springer, Berlin, 1998) pp.
	%1–12.

%more topics
  %robotic map making literature
  %spanning forests
  %A*
  %Priority Search Queues
  %Hamiltonian Cycle