\title{Path Planning for Variable Scrutiny Multi-Robot Coverage}
%\title{Online Planning for Robotic Coverage}

\author{Kevin Bradner\\Department of Electrical Engineering and Computer Science\\Case Western Reserve University}
% \date{\monthyeardate\today} %date the document was created
\date{January, 2020}
%\maketitle

\begin{center}
\begin{huge}
\textbf{PATH PLANNING FOR VARIABLE SCRUTINY MULTI-ROBOT COVERAGE}

\parskip 0.5 in

\textbf{by}

\parskip 0.75 in

\textbf{KEVIN BRADNER}
\end{huge}

\begin{Large}

\parskip 0.5 in

Submitted in partial fulfillment of the requirements for the degree of Master of Science

Department of Electrical Engineering and Computer Science

\textbf{CASE WESTERN RESERVE UNIVERSITY}

May, 2020
\end{Large}
\end{center}

\clearpage

\begin{center}
\begin{Large}
\textbf{CASE WESTERN RESERVE UNIVERSITY}
\linebreak
\textbf{SCHOOL OF GRADUATE STUDIES}
\end{Large}

\parskip 0.5 in

We hereby approve the thesis of

\parskip 0.05 in

\begin{Large}
Kevin Bradner
\end{Large}

\parskip 0.4 in

\parskip 0.05 in

candidate for the degree of \textbf{Master of Science*}.

\parskip 0.5 in

Committee Chair
\linebreak
\textbf{Dr. Wyatt Newman}

\parskip 0.5 in

Committee Member
\linebreak
\textbf{Dr. Soumya Ray}

\parskip 0.5 in

Committee Member
\linebreak
\textbf{Dr. M. Cenk {\c C}avu{\c s}o{\u g}lu}

\parskip 0.5 in

Date of Defense
\linebreak
\textbf{January 9, 2020}

\parskip 0.5 in

*We also certify that written approval has been obtained
\linebreak
for any proprietary material contained therein.

\end{center}


\tableofcontents

\listoftables

\listoffigures

\pagebreak

\begin{center}

Path Planning for Variable Scrutiny Multi-Robot Coverage

Abstract

by

KEVIN BRADNER

\end{center}

Robotic coverage problems task one or more robots with the goal of visiting every location in a region. Algorithms that can perform this kind of task in a time-efficient manner are useful for purposes such as mapping, cleaning, or inspection. This work considers a multi-agent robotic coverage problem in which the shape of the region to be explored is known in advance, but additional information and challenges are discovered during task performance. A software package is created to simulate such a scenario, generate virtual environments to be covered, and interface with policy programs that command the robots. Simple path planning algorithms are developed, and their performance on this task is evaluated. Next, adaptive variants of these algorithms are developed to respond to events and information encountered during task execution. It is shown that the adaptive algorithms are more robust and better performing than their more basic counterparts.