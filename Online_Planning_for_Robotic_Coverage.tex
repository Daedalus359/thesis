\documentclass[letterpaper, 12pt, leqno]{report}

\usepackage[left=1.5in,right=1.5in,top=1.0in,bottom=1.0in]{geometry}
\usepackage{setspace}
\usepackage{datetime}
\usepackage{mathptmx}

\newdateformat{monthyeardate}{%
  \monthname[\THEMONTH], \THEYEAR}

\doublespacing

\fontfamily{ptm}\selectfont

\begin{document}

\pagenumbering{roman}

\title{Variable Detail Robotic Sensor Coverage}
%\title{Online Planning for Robotic Coverage}

\author{Kevin Bradner\\Department of Electrical Engineering and Computer Science\\Case Western Reserve University}
% \date{\monthyeardate\today} %date the document was created
\date{January, 2020}
\maketitle

\tableofcontents

\listoftables

\listoffigures

\section*{Acknowledgements}

\begin{abstract}
Robotic coverage problems task one or more robots with the goal of visiting every location in a region. Algorithms which can perform this kind of task in a time efficient manner are useful for purposes such as mapping, cleaning, or inspection. This work considers a multi agent robotic coverage problem in which the shape of the region to be explored is known in advance, but additional information and challenges are discovered during task performance. A software package is created to simulate such a scenario, generate virtual environments to be covered, and interface with policy programs that command the robots. Offline path planning algorithms are developed, and their performance on this task is evaluated. Next, online variants of these algorithms are developed to respond to events and information encountered during task execution. It is shown that the online algorithms are more robust and better performing than their offline counterparts.
\end{abstract}

\pagenumbering{arabic}

\chapter{Introduction}
% 20 pages or so
% rules of the game
% objective of my algorithms
% literature review

\section{Robotic Coverage Background}

Robotic motion planning problems are typically concerned with finding an achievable path between two robot configurations. These configurations could be the current location and desired destination of a robotic vehicle, the current pose of a robotic arm and the pose required for it to grasp an object, or any other pair of robot states which correspond to ``start" and ``goal" configurations. \textit{Robotic coverage} problems are another kind of motion planning problem. The goal of a robotic coverage problem is to find a path for the robot to follow, during which it will visit every location in a region. This region could be an area of floor to be vacuumed, the surface of a car body that needs to be painted, or the interior of a building that needs to be mapped. In all of these cases, the task is complete only once the entire region has been visited, or \textit{covered} by a robot. Usually, another goal of these problems is to achieve coverage in a way that minimizes the time or number of movements required to complete the task.
The robotic coverage problem can be viewed as a continuous generalization of the travelling salesman problem [7]. Thus, the problem is NP-Hard. 

% based on [1], section 1. Is this too similar to the original?
One particularly well studied type of robotic coverage problem is the case where the coverage region is some subset of the plane, and the robots are mobile robots that can only move in the plane. A good overview of this kind of coverage problem can be found in [1]. Algorithms to perform robot coverage in the plane have been studied since the late 1980's. Early algorithms for this purpose relied on the use of heuristics, and these algorithms usually could not guarantee that complete coverage would always be achieved. By around 2000, planar coverage algorithms were often able to guarantee complete coverage. A completeness guarantee is often proved by showing that the algorithm breaks down the coverage space into smaller regions that are relatively trivial to cover completely. Coverage algorithms that employ this kind of technique are known as \textit{cellular decomposition} algorithms.

One type of cellular decomposition fits a square grid to the coverage space. Because arbitrary sets in the plane can not be exactly decomposed into square regions in a general case, this technique is often called approximate cellular decomposition. Typically, each cell in this kind of decomposition can be covered entirely by a single robot configuration. For example, consider dividing a grass lawn into squares for a robotic lawn mowing task. If each square in the cellular decomposition is small enough, a lawnmower whose center is aligned with the square's center will have mowed that entire square of the lawn. In coverage problems where this assumption applies, proving that a coverage path visits each cell in the approximate decomposition may be enough to guarantee complete coverage.

In addition to approximate cellular decomposition, there are also techniques that decompose a coverage space into a more diverse collection of cell shapes. Examples of such shapes include cells of fixed with whose top and bottom boundaries can vary in shape [3], or trapezoidal cells whose parallel sides all run in the same direction [4]. Depending on whether the union of these cells is exactly equal to the coverage space, such techniques are called either \textit{semi-approximate} or \textit{exact} cell decompositions. In the case of trapezoidal cell decomposition, it is possible to combine the trapezoidal cells into a smaller number of large cells, each of which are able to be covered by a simple back-and-forth sweeping motion. This approach is known as a \textit{boustrophedon} decomposition [4].

\section{Spanning-Tree based Coverage Algorithms}

When coverage is performed by a square shaped tool attached to a mobile robot, and when the coverage area can be exactly decomposed into a grid of squares which have double the side length of the robot tool, it is possible to compute an optimal covering path in linear time using an approach based on spanning trees [2]. The authors developed several variants of this approach with different time, memory, and prior knowledge requirements. In all cases, it was assumed that a mobile robot would complete the coverage task with a square shaped tool of size \textit{D}. It was also assumed that the tool could only move in the four cardinal directions when completing the task. Finally, it was assumed that the region to be covered could approximated by cells in a grid of squares with size 2\textit{D}. The authors argue that the final assumption is justified when the size of the robots tool is significantly smaller than the dimensions of the area to be covered. Under these assumptions, the \textit{Spanning Tree Covering} (STC) algorithm is able to generate a path through the space which visits each grid-aligned square cell of size D exactly once.

The offline variant of STC works by representing the coverage area as a graph \textit{G}. Nodes on this graph represent the center of a square with size 2\textit{D}, and the set of nodes corresponds to the set of full cells when a grid with square size 2\textit{D} is overlaid on the coverage space. Edges exist between any two nodes whose corresponding cells are adjacent in the coverage space. As the name suggests, STC's key step is the creation of a spanning tree on \textit{G}, using any node in \textit{G} as the root of the tree. Many algorithms exist to compute spanning trees, but two popular examples are closely based on breadth-first-search and depth-first-search [9].

%insert a picture here

\section{Muti-Agent Robotic Coverage}

Another imporant feature of a robotic coverage algorithm is whether that algorithm controls a single robot or a group. Algorithms that control a group of robots for this purpose are known as multi-agent coverage algorithms. Multi agent approaches to coverage can often complete a coverage task several times faster than a single agent approach because of the division of work that use of multiple robots enables. Use of multiple robots can enhance robotic coverage by allowing for more precise localization through information sharing. Finally, multi-agent methods are often able to adapt to the failure of one or more robots, as there may be remaining robots capable of completing the coverage task [1]. However, multi-agent techniques often require more complex motion planning strategies to coordinate the motion of multiple robots in a way that takes full advantage of these potential efficiency improvements and other benefits.

It is also possible to adapt a spanning tree coverage algorithm to the multi-agent case [6].

\section{Problem Statement}

This work presents multi-agent coverage algorithms based on spanning trees.

%constructing a hamiltonian cycle in a general grid-like graph is NP-complete, see citation 19 from spanning tree covering (my [2])

\chapter{Simulation Development}

\chapter{Offline Planning Policies}

\chapter{Online Planning Policies}

\chapter{Conclusion}
% summary and directions for future work

\chapter{Bibliography}

% [#], Author, "Title," \textit{publication name,} vol. ##, PubMonth, PubYear.

\noindent [1] H. Choset, ``Coverage for robotics - A survey of recent results,'' \textit{Annals of \mbox{Mathematics} and Artificial Intelligence,} vol. 31, October 2001.
%Issue 1-4, pp 113-126
	% complete a priori environment information or sensir based coverage, see section 1
	% random approaches, see section 2
	% potential field methods, sec 3
	% semi-approximate cellular decomposition, sec 4

\noindent [2] Y. Gabriely and E. Rimon, ``Spanning-tree based coverage of continuous areas by a mobile robot,'' \textit{Annals of \mbox{Mathematics} and Artificial Intelligence,} vol. 31, October 2001.
%Issue 1-4, pp 77-98

\noindent [3] S. Hert, S. Tiwari, and V. Lumelsky, ``A terrain-covering algorithm for an AUV,'' \textit{Autonomous Robots,} vol. 3, June 1996.
%Issue 2-3, pp 91-119
	%an algorithm that can be performed on-line

\noindent [4] H. Choset, E. Acar, A. Rizzi, and J. Luntz, ``Exact Cellular Decomposition in Terms of Critical Points of Morse Functions,'' \textit{Proceedings of the 2000 IEEE International Conference on Robotics and Automation,} April 2000.

\noindent [5] D. Kurabayashi, J. Ota, T. Arai, S. Ichikawa, S. Koga, H. Asama, and I. Endo, ``Cooperative Sweeping by Multiple Mobile Robots with Relocating Portable Obstacles," \textit{Proceedings of the IEEE/RSJ International Conference on Intelligent Robots and Systems,} November 1996

\noindent [6] X. Zheng, S. Jain, S. Koenig, and D. Kempe, ``Multi-Robot Forest Coverage," \textit{IEEE/RSJ International Conference on Intelligent Robots and Systems,} August 2005

%alternate citation which may prove what I need, since I can't find a copy of this one:
  %E.M. Arkin and R. Hassin, Approximation algorithms for the geometric covering salesman problem, Discrete Appl. Math. 55 (1994) 197–218.
\noindent [7] E. Arkin, S. Fekete, and J. Mitchell, ``The lawnmower problem," \textit{Proceedings of the 5th Canadian Conference on Computational Geometry,} August 1993
% pp 461-466

\noindent [8] Y. Gabriely and E. Rimon, ``On-Line Coverage of Grid Environments by a Mobile Robot," \textit{Computational Geometry,} April 2003
% pp 197-224

%introduction to algorithms BOOK
\noindent [9] T. Cormen, C. Lieserson, R. Rivest, and C. Stein, \textit{Introduction to Algorithms, Second Edition,} MIT Press and McGraw-Hill, 2001


%More references to use
  %1
	%S.V. Spires and S.Y. Goldsmith, Exhaustive geographic search with mobile robots along space-filling
	%curves, in: Collective Robotics; Proceedings of the First International Workshop, CRW’98. Paris,
	%France, July 1998, Lecture Notes in Artificial Intelligence, Vol. 1456 (Springer, Berlin, 1998) pp.
	%1–12.
  %2
    %

%other ideas
  %look into the robotic map making literature

%consult the list of topics I made as a background list on github

%spanning trees and the algorithms used to create them


\end{document}	