\documentclass[letterpaper, 12pt, leqno]{report}

\usepackage[left=1.5in,right=1.5in,top=1.0in,bottom=1.0in]{geometry}
\usepackage{setspace}
\usepackage{datetime}
\usepackage{mathptmx}

\newdateformat{monthyeardate}{%
  \monthname[\THEMONTH], \THEYEAR}

\doublespacing

\fontfamily{ptm}\selectfont

\begin{document}

\pagenumbering{roman}

\title{Online Planning for Robotic Coverage}
\author{Kevin Bradner\\Department of Electrical Engineering and Computer Science\\Case Western Reserve University}
% \date{\monthyeardate\today} %date the document was created
\date{January, 2020}
\maketitle

\tableofcontents

\listoftables

\listoffigures

\section*{Acknowledgements}

\begin{abstract}
Robotic coverage problems task one or more robots with the goal of visiting every location in a region. Algorithms which can perform this kind of task in a time efficient manner are useful for purposes such as mapping, cleaning, or inspection. This work considers a multi agent robotic coverage problem in which the shape of the region to be explored is known in advance, but additional information and challenges are discovered during task performance. A software package is created to simulate such a scenario, generate virtual environments to be covered, and interface with policy programs that command the robots. Offline path planning algorithms are developed, and their performance on this task is evaluated. Next, online variants of these algorithms are developed to respond to events and information encountered during task execution. It is shown that the online algorithms are more robust and better performing than their offline counterparts.
\end{abstract}

\pagenumbering{arabic}

\chapter{Problem Statement}
% 20 pages or so
% rules of the game
% objective of my algorithms
% literature review

\section{Background}

\chapter{Simulation Deveopment}

\chapter{Offline Planning Policies}

\chapter{Online Planning Policies}

\chapter{Conclusion}
% summary and directions for future work

\end{document}	