\documentclass[letterpaper, 12pt, leqno]{report}

\usepackage[left=1.5in,right=1.5in,top=1.0in,bottom=1.0in]{geometry}
\usepackage{setspace}
\usepackage{datetime}
\usepackage{mathptmx}

\newdateformat{monthyeardate}{%
  \monthname[\THEMONTH], \THEYEAR}

\doublespacing

\fontfamily{ptm}\selectfont

\begin{document}

\pagenumbering{roman}

\title{Online Planning for Robotic Coverage}
\author{Kevin Bradner\\Department of Electrical Engineering and Computer Science\\Case Western Reserve University}
% \date{\monthyeardate\today} %date the document was created
\date{January, 2020}
\maketitle

\tableofcontents

\listoftables

\listoffigures

\section*{Acknowledgements}

\begin{abstract}
Robotic coverage problems task one or more robots with the goal of visiting every location in a region. Algorithms which can perform this kind of task in a time efficient manner are useful for purposes such as mapping, cleaning, or inspection. This work considers a multi agent robotic coverage problem in which the shape of the region to be explored is known in advance, but additional information and challenges are discovered during task performance. A software package is created to simulate such a scenario, generate virtual environments to be covered, and interface with policy programs that command the robots. Offline path planning algorithms are developed, and their performance on this task is evaluated. Next, online variants of these algorithms are developed to respond to events and information encountered during task execution. It is shown that the online algorithms are more robust and better performing than their offline counterparts.
\end{abstract}

\pagenumbering{arabic}

\chapter{Introduction}
% 20 pages or so
% rules of the game
% objective of my algorithms
% literature review

\section{Background}

Robotic motion planning problems are typically concerned with finding an achievable path between two robot configurations. These configurations could be the current location and desired destination of a robotic vehicle, the current pose of a robotic arm and the pose required for it to grasp an object, or any other pair of robot states which correspond to "start" and "goal" configurations. \textit{Robotic coverage} problems are another kind of motion planning problem. The goal of a robotic coverage problem is to find a path for the robot to follow, during which it will visit every location in a region. This region could be an area of floor to be vacuumed, the surface of a car body that needs to be painted, or the interior of a building that needs to be mapped. In all of these cases, the task is complete only once the entire region has been visited, or \textit{covered} by a robot. Usually, another goal of these problems is to achieve coverage in a way that minimizes the time or number of movements required to complete the task.

% based on [1], section 1. Is this too similar to the original?
One particularly well studied type of robotic coverage problem is the case where the coverage region is some subset of the plane, and where the robots are mobile robots that can only move in the plane. A good overview of this kind of coverage problem can be found in [1]. Algorithms to perform robot coverage in the plane have been studied since the late 1980's. Early algorithms for this purpose relied on the use of heuristics, and these algorithms usually could not guarantee that complete coverage would always be achieved. By around 2000, planar coverage algorithms were often able to guarantee complete coverage. A completeness guarantee is often proved by showing that the algorithm breaks down the coverage space into smaller regions that are relatively trivial to cover completely. Coverage algorithms that employ this kind of technique are known as \textit{cellular decomposition} algorithms.

One type of cellular decomposition fits a square grid to the coverage space. Because arbitrary sets in the plane can not be exactly decomposed into square regions in a general case, this technique is often called approximate cellular decomposition. Typically, each cell in this kind of decomposition can be covered entirely by a single robot configuration. For example, consider dividing a grass lawn into squares for a robotic lawn mowing task. If each square in the cellular decomposition is small enough, a lawnmower whose center is aligned with the square's center will have mowed that entire square of the lawn. In coverage problems where this assumption applies, proving that a coverage path visits each cell in the approximate decomposition may be enough to guarantee complete coverage.


\section{Problem Statement}

\chapter{Simulation Development}

\chapter{Offline Planning Policies}

\chapter{Online Planning Policies}

\chapter{Conclusion}
% summary and directions for future work

\chapter{Bibliography}

% [#], Author, "Title," \textit{publication name,} vol. ##, PubMonth, PubYear.

\noindent [1] H. Choset, ``Coverage for robotics - A survey of recent results,'' \textit{Annals of \mbox{Mathematics} and Artificial Intelligence,} vol. 31, October 2001.
%Issue 1-4, pp 113-126
	% complete a priori environment information or sensir based coverage, see section 1
	% random approaches, see section 2
	% potential field methods, sec 3

\noindent [2] Y. Gabriely and E. Rimon, ``Spanning-tree based coverage of continuous areas by a mobile robot,'' \textit{Annals of \mbox{Mathematics} and Artificial Intelligence,} vol. 31, October 2001.
%Issue 1-4, pp 77-98


\end{document}	